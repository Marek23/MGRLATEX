\newpage
\begin{center}
\large \bf
POPRAWA JAKOŚCI OBRAZÓW CYFROWYCH POPRZEZ METODĘ WMALOWYWANIA
\end{center}
\section*{Streszczenie}
W pracy dokonano przeglądu istniejących algorytmów wmalowywania opartych na rozwiązaniu odpowiednio zdefiniowanych problemów matematycznych. Wyniki badanych zagadnień uzyskiwane są na podstawie danych wejściowych w postaci zdefiniowanych obszarów obrazu otaczających brakujące regiony. Podstawowymi narzędziami wykorzystywanymi w algorytmach rekonstrukcji są cząstkowe równania różniczkowe wyższego rzędu oraz definicje metryk będące miarą podobieństwa badanych obszarów. W niniejszej pracy opisano wybrane algorytmy, podano sposoby dyskretyzacji, zaimplementowano, przetestowano, opisano dobór ich optymalnych parametrów i ostatecznie porównano je ze sobą. Zaproponowano nowe modele bądź ulepszenia do zaimplementowanych algorytmów uzupełniania brakujących danych. Zwrócono także uwagę na różne cechy obrazu, które mogą determinować dobór algorytmu do zadanego problemu. \textbf{Do implementacji prezentowanych przez autora algorytmów skorzystano z: Matlab'a (narzędzia mex), języka C, C++ oraz Python.}

\bigskip
{\noindent \bf Słowa kluczowe:} Inpainting, wmalowywanie, przetwarzanie obrazów

\vskip 2cm

\begin{center}
\large \bf
IMPROVEMENT QUALITY OF DIGITAL IMAGE THROUGHT THE INPAINTING METHOD
\end{center}

\section*{Abstract}
Existing inpainting algorithms based on solving properly defined mathematical problems were reviewed. The results of the studied issues are obtained on the basis of input data in the form of defined image areas surrounding the missing regions. The basic tools used in reconstruction algorithms are higher-order partial differential equations and definitions of metrics that measure the similarity of the image patches. This work describes selected algorithms, their discretization methods, types of implementation, tests, the selection of their optimal parameters and finally compare to each other. New models or improvements have been proposed to the implemented inpainting algorithms. Attention was also paid to various image features that may determine the selection of an algorithm for a given problem. \textbf {Implementation of described algorithms where made using: Matlab (mex tool), C, C ++ and Python.}

\bigskip
{\noindent \bf Keywords:} Inpainting, image processing

\vfill