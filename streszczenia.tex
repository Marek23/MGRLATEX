\newpage
\begin{center}
\large \bf
POPRAWA JAKOŚCI OBRAZÓW CYFROWYCH POPRZEZ METODĘ WMALOWYWANIA
\end{center}
\section*{Streszczenie}
Głównym celem tej pracy magisterskiej jest opracowanie algorytmu skutecznie odbudowującego wyznaczony obszar w obrazie. W tym celu dokonano przeglądu istniejących algorytmów wmalowywania opartych na rozwiązaniu odpowiednio zdefiniowanych problemów matematycznych. Wyniki badanych zagadnień uzyskiwane są na podstawie danych wejściowych w postaci zdefiniowanych obszarów obrazu otaczających brakujące regiony. Podstawowymi narzędziami wykorzystywanymi w algorytmach rekonstrukcji są cząstkowe równania różniczkowe wyższego rzędu oraz definicje metryk będące miarą podobieństwa badanych obszarów. W niniejszej pracy opisano wybrane algorytmy, podano sposoby dyskretyzacji, zaimplementowano, przetestowano, opisano dobór ich optymalnych parametrów i ostatecznie porównano je ze sobą. Zaproponowano nowe modele bądź ulepszenia do zaimplementowanych algorytmów uzupełniania brakujących danych. Zwrócono także uwagę na różne cechy obrazu, które mogą determinować dobór algorytmu do zadanego problemu. W dziale pierwszym autor tej pracy magisterskiej wprowadza czytelnika do zagadnienia przetwarzania obrazów. W dziale drugim dokonuje przeglądu istniejących algorytmów. W dziale trzecim autor tej pracy umieszcza dokładny opis metody bazującej  na rozwiązaniu równania Naviera-Stokesa. W działach kolejno czwartym i piątym przedstawia opis kolejno trzech rozwiązań: $NLCTV$,  $VFI$ oraz metodę Criminisi, zaliczających się do grupy nielokalnego przetwarzania obrazów. W dziale szóstym autor tej pracy magisterskiej przedstawia własne propozycje rozwiązań i modyfikacji. W dziale siódmym autor przedstawia i porównuje ze sobą wyniki uzyskane na podstawie rozwiązań przedstawionych w tej pracy magisterskiej. \textbf{Do implementacji prezentowanych przez autora algorytmów skorzystano z: Matlab'a (narzędzia mex), języka C, C++ oraz Python.}


\bigskip
{\noindent \bf Słowa kluczowe:} Inpainting, wmalowywanie, przetwarzanie obrazów

\vskip 2cm

\begin{center}
\large \bf
IMPROVING THE QUALITY OF DIGITAL IMAGES THROUGHT THE INPAINTING METHOD
\end{center}

\section*{Abstract}
The main purpose of this master's thesis is to develop an algorithm that effectively rebuilds the missing area in the image. For this purpose, the existing inpainting algorithms based on solving properly defined mathematical problems were reviewed. The results of the studied issues are obtained on the basis of input data in the form of defined image areas surrounding the missing regions. The basic tools used in reconstruction algorithms are higher-order partial differential equations and definitions of metrics that measure the similarity of the image patches. This work describes selected algorithms, their discretization methods, types of implementation, tests, the selection of their optimal parameters and finally compare to each other. New models or improvements have been proposed to the implemented inpainting algorithms. Attention was also paid to various image features that may determine the selection of an algorithm for a given problem. In the first section, the author of this master's thesis introduces the reader to the issue of image processing. In the second section, he reviews existing algorithms. In the third chapter, the author provides a detailed description of the method based on the solution of the Navier-Stokes equation. Fourth and fifth chapters includes a description of three solutions: $NLCTV$, $VFI$ and Criminisi method, all belonging to the group of non-local image processing. In the sixth chapter, the author of this master's thesis presents his own solutions and modifications. In chapter seven, the author shows and compares inpainting results received by presented methods. \textbf {Implementation of described algorithms where made using: Matlab (mex tool), C, C ++ and Python.}

\bigskip
{\noindent \bf Keywords:} Inpainting, image processing

\vfill